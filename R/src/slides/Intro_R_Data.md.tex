\begin{frame}{R?}

R (pronounced aRrgh -- pirate style) is a programming language and
environment for statistical computing and graphics

\begin{itemize}
\item
  oriented towards data handling analysis and storage facility
\item
  R Base
\item
  Packages tools and functions (user contributed)
\item
  R Base and most R packages are available from the
  \href{cran.r-project.org}{Comprehensive R Archive Network (CRAN)}
\item
  Use R console or IDE: \textbf{Rstudio}, Deducer, vim/emacs\ldots{}
\item
  Comment is \textbf{\#}, help is \textbf{?} before a function name
\end{itemize}

\end{frame}

\begin{frame}[fragile]{Using R}

\begin{block}{\textbf{Installing/using packages}}

Install and load the \texttt{ggplot2} package (even if already
installed)

\begin{verbatim}
install.packages("ggplot2")
library(ggplot2)
\end{verbatim}

Or in one step, install if not available then load:

\begin{verbatim}
require(ggplot2) || {install.packages("ggplot2");
                     require(ggplot2)}
\end{verbatim}

\end{block}

\end{frame}

\begin{frame}{Using R}

\begin{block}{\textbf{Usefull Functions}}

\begin{itemize}
\item
  List all objects in memory: \texttt{ls()}
\item
  Save an object: \texttt{save(obj,\ file)}
\item
  Load an object: \texttt{load(file)}
\item
  Set working directory: \texttt{setwd(dir)}
\end{itemize}

\end{block}

\end{frame}

\begin{frame}{Data Structures}

\begin{itemize}
\item
  scalar:

  s = 3.14
\item
  vector:

  v = c(1, 2, ``ron'')
\item
  list:

  l = list(1:10, `a', pi)
\item
  matrix:

  m = matrix(seq(1:6), 2)
\item
  \textbf{dataframe}:

  df = data.frame(``col1'' = seq(1:4), ``col2'' = c(5, 6, ``cerveza'',
  6*7))
\item
  \ldots{}
\end{itemize}

\end{frame}

\begin{frame}[fragile]{Entering Data}

\begin{block}{Reading CSV or text files}

\begin{verbatim}
# comma separated values
dat.csv <- read.csv(<file or url>)
# tab separated values
dat.tab <- read.table(<file or url>, 
    header=TRUE, sep = "\t")
\end{verbatim}

\end{block}

\end{frame}

\begin{frame}[fragile]{Entering Data}

\begin{block}{Reading data from other software: Excel, SPSS\ldots{}}

Excel Spreadsheets -- need \texttt{xlsx} package

\begin{verbatim}
read.xlsx()
\end{verbatim}

SPSS and Stata both need the \texttt{foreign} package

\begin{verbatim}
dat.spss <- read.spss(<file or url>, 
                      to.data.frame=TRUE)
             
dat.dta <- read.dta(<file or url>)
\end{verbatim}

\end{block}

\end{frame}

\begin{frame}{Data Frames}

Most easy structure to use, have a matrix structure.

\begin{itemize}
\item
  \textbf{Observations} are arranged as \textbf{rows} and
  \textbf{variables}, either numerical or categorical, are arranged as
  \textbf{columns}.
\item
  Individual rows, columns, and cells in a data frame can be accessed
  through many methods of indexing.
\item
  We most commonly use \textbf{object{[}row,column{]}} notation.
\end{itemize}

\end{frame}

\begin{frame}[fragile]{Accessing Items in a \texttt{data.frame}}

Aside with R are provided example datasets, i.e. \texttt{mtcars} that
can be used

\begin{verbatim}
data(mtcars)
head(mtcars)
colnames(mtcars)

# single cell value
mtcars[2,3]
# omitting row value implies all rows
mtcars[,3]
# omitting column values implies all columns
mtcars[2,]
\end{verbatim}

\end{frame}

\begin{frame}[fragile]{Accessing Items in a \texttt{data.frame}}

We can also access variables directly by using their names, either with
\textbf{object{[},``variable''{]}} notation or \textbf{object\$variable}
notation.

\begin{verbatim}
# get first 10 rows of variable `mpg` using two methods:
mtcars[1:10, "mpg"]
mtcars$mpg[1:10]
\end{verbatim}

\end{frame}

\begin{frame}{Exploring Data}

\begin{block}{Description Of Dataset}

\begin{itemize}
\item
  Using \textbf{dim}, we get the number of observations(rows) and
  variables(columns) in the dataset.
\item
  Using \textbf{str}, we get the structure of the dataset, including the
  class(type) of all variables.

  dim(mtcars) str(mtcars)
\item
  \textbf{summary} when used on a dataset, returns distributional
  summaries of variables in the dataset.

  summary(mtcars)
\item
  \textbf{quantile} function enables to get statistical metrics on the
  selected data

  quantile(mtcars\$mpg)
\end{itemize}

\end{block}

\end{frame}

\begin{frame}{Exploring Data}

\begin{block}{Conditional Exploration}

\begin{itemize}
\item
  \textbf{subset} enables to explore data conditionally

  subset(mtcars, cyl \textless{}= 5)
\item
  \textbf{by} enables to call a particular function to sub-groups of
  data

  by(mtcars, mtcars\$cyl, summary)
\end{itemize}

\end{block}

\end{frame}

\section{Crazy Examples}\label{crazy-examples}

\begin{frame}[fragile]{Random Text Generation}

\begin{verbatim}
library(XML)
stem <- "http://www.5novels.com/classics/u5688"
hobbit <- NULL
for(i in 1:74) {
    if(i==1) { url <- paste0(stem, ".html") } 
    else { url <- paste0(stem, "_", i, ".html") }
    x <- htmlTreeParse(url, useInternalNodes=TRUE)
    xx <- xpathApply(x, "//p", xmlValue)
    hobbit <- c(hobbit, gsub("\r", "", xx[-length(xx)]))
    Sys.sleep(0.5)
}
hobbit = paste(hobbit, collapse=' ')
\end{verbatim}

\end{frame}

\begin{frame}[fragile]{Random Text Generation -- 2}

\begin{verbatim}
library(ngram)
ng2 <- ngram(hobbit, n=2)

babble(ng2, 128, seed=987654)
\end{verbatim}

\end{frame}
